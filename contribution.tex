
\begin{abstract}
 According to many forecasts, the trend of information technology is going to an integration of information processing into a kind of physical environment. Hence, we need a robust security mechanism to secure Cyber-Physical Systems (CPS). In this paper we provide the state-of-the-art work regarding security in CPS systems, identifying CPS security objectives, threat and related attacks, we also investigate CPS security challenges. Moreover, we survey the security challenges in different application areas.
\end{abstract}
\section{Introduction To CPS}
Cyber-physical systems (CPS) are smart systems that include engineered interacting networks of physical and computational components. Bringing these different components together aims to improve the quality of modern life and introducing the advances of technology to critical areas. The impact of CPS in life can be spotted in many areas, such as; autonomous vehicles, intelligent buildings, smart energy systems, robots, and smart medical devices. Figure 1 shows a general architecture for CPSs, where computation connected with physical processes through networks.

\begin{figure}[h!]
  \includegraphics[width=\linewidth]{arch.png}
  \caption{Proposed design.}
  \label{fig:design}
\end{figure} 

\subsection{Trustworthiness in CPS}
National Institute of Standards and Technology (NIST) defines trustworthiness of cyber-physical systems as the demonstrable likelihood that the system performs according to designed behaviour under any set of conditions as evidenced by characteristics including, but not limited to, safety, security, privacy, reliability and resilience.

\section{Introduction to security issues in CPS}
Designing and implementing a security solution for securing a CPS is a challenging task due to dual nature of the CPS (cyber and physical). Any solution must consider both of them in integrated manner. Many challenges involved in preventing, detecting and mitigating the attacks.
\subsection{Differences between cyber-security and CPS security} 

\section{Threat models for CPS}
A systematic study of the security of any system requires a proper description of possible threats, building such a treat model helps understanding the scope of the problem and assess the risks. A majority of existing approaches for threat modelling can be broadly divided into two main groups – attack tree-based approaches and stochastic model-based approaches.
This section outlines some of threat modelling techniques that  have  been  applied  to  the  CPS  domain

\section{Conclusion} 
 
 

\begin{thebibliography}{9}

 
\bibitem{einstein} 
Giraldo, J., Sarkar, E., Cardenas, A. A., Maniatakos, M., & Kantarcioglu, M.\textit{Security and privacy in cyber-physical systems: A survey of surveys. }. 
IEEE Design & Test, 34(4), 7-17., 2017.
 
\bibitem{knuthwebsite} 
  Marwedel, P.
\textit{ Embedded System Foundations of Cyber-Physical Systems.}. 
2011.

\bibitem{knuthwebsite} 
  Ali, S., Qaisar, S. B., Saeed, H., Khan, M. F., Naeem, M., & Anpalagan, A.
\textit{Network challenges for cyber physical systems with tiny wireless devices: A case study on reliable pipeline condition monitoring.}.
Sensors, 15(4), 7172-7205,2015.

\end{thebibliography}


